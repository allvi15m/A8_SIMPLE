For testing the algorithm described in section \ref{A*} a Florida feeder available in the SUNGRIN project \cite{SUNGRIN} is used. Fig. \ref{fig:simulation_grid} shows the whole system used for testing the algorithm. The yellow borders marks the section of the feeder modified to construct a microgrid similar to the one shown in Fig \ref{fig:system_arch}. The PV and load inside the microgrid is modeled using the load and solar data collected form the SUNGRIN project. An energy storage is included to construct the microgrid. Table \ref{tab:solar_pv} shows the physical parameters of the PV plant and its inverter. The physical parameters of the energy storage used are shown in table \ref{tab:es}. The levelized cost of energy (LCOE)  of the energy storage system $R_{ESS}$ is calculated using equation \ref{eq:R_ESS}.

\begin{figure}[!ht]
    \centering
    \includegraphics[width = \linewidth]{figs/simulation_grid.png}
    \caption{Simulated system}
    \label{fig:simulation_grid}
\end{figure}


\begin{equation}
\label{eq:R_ESS}
R_{ESS} = \dfrac{ES_{tot}}{Cyc\cdot ES_{Cap}\cdot DoD\cdot \eta_{r}},
\end{equation}

%%%%%%%%PV%%%%%%%%%%%%%%%%%%%%%%%%%%%%%%%%%%%
\begin{table}[htb]
%\normalsize
%\renewcommand{\arraystretch}{1}
\caption{PV System Specifications}
\label{tab:solar_pv}
\centering
    \begin{tabular}{ | l | p{3cm} | }
    \hline
    \textbf{PV System Parameters} & \textbf{Value} \\ \hline
    PV Panels Rating (\(P_{PV}\)) & 875 kW  \\ \hline
    Inverter Rating (\(S_{PV}\)) & 900 kVA \\ \hline
    Power Factor Range (\(pf_{PV}\)) & 0.8-1.0  \\ \hline
    Max. Reactive Power (\(\overline{Q_{PV}}\)) & 540 kVAR \\ \hline
    Min. Reactive Power (\(\underline{Q_{PV}}\)) & -540 kVAR \\ \hline
    LCOE (\(r_{PV}\)) & 2.51 c/kWh \\ \hline
    \end{tabular}
    \begin{tabular}{l}
    \end{tabular}
\end{table}
%%%%%%%%PV%%%%%%%%%%%%%%%%%%%%%%%%%%%%%%%%%%%


%%%%%%%%ES%%%%%%%%%%%%%%%%%%%%%%%%%%%%%%%%%%%
\begin{table}[htb]
%\normalsize
%\renewcommand{\arraystretch}{1}
\caption{Energy Storage (ES) System Specifications}
\label{tab:es}
\centering
    \begin{tabular}{ | l | p{3cm} | p{3cm} | }
    \hline
    \textbf{ES System Parameters} & \textbf{Value} \\ \hline
    ES Rating (\(P_{ES}\)) & 750 kW  \\ \hline
    Inverter Rating (\(S_{ES}\)) & 750 kVA \\ \hline
    Max. State of Charge  (\(\overline{SOC_{ES}}\)) & 2190 kWh \\ \hline
    Min. State of Charge  (\(\underline{SOC_{ES}}\)) & 219 kWh \\ \hline
    Power Factor Range (\(pf_{ES}\)) & 0.8-1.0  \\ \hline
    Max. Reactive Power (\(\overline{Q_{ES}}\)) & 450 kVAR \\ \hline
    Min. Reactive Power (\(\underline{Q_{ES}}\)) & -450 kVAR \\ \hline
    LCOE (\(r_{ES}\)) & 12.3 c/kWh \\ \hline
    \end{tabular}
\end{table}
%%%%%%%%ES%%%%%%%%%%%%%%%%%%%%%%%%%%%%%%%%%%%

Fig \ref{fig:LOAD_PROFILE_8} shows the load profile of the system for eight days with average, minimum and maximum loads.

\begin{figure}[!ht]
    \centering
    \includegraphics[width = \linewidth]{figs/loadprofile.png}
    \caption{Eight day load profile}
    \label{fig:LOAD_PROFILE_8}
\end{figure}

To generate the RTP profile Locational Based Marginal Pricing (LBMP) data is collected from the New York Independent System Operator (NYISO) \cite{NYISO2017}. The collected LBMPO is combined with the time of use price available at Tallahassee to generate a RTP for the test case. Fig \ref{fig:RTP_PROFILE_8} shows the real-time price profile used for the test system. The system is also tested against time of use pricing scheme available from PG\&E \cite{pgne}. Fig \ref{fig:PGNE_PRICE} shows the price profile acquired from PG\&E.

\begin{figure}[!ht]
    \centering
    \includegraphics[width = \linewidth]{figs/rtp_8days.png}
    \caption{Eight day RTP profile NYISO}
    \label{fig:RTP_PROFILE_8}
\end{figure}

\begin{figure}[!ht]
    \centering
    \includegraphics[width = \linewidth]{figs/PGNE_PRICE.png}
    \caption{Eight day RTP profile PG\&E}
    \label{fig:PGNE_PRICE}
\end{figure}

The PV profile used in the system is collected from the SUNGRIN project and scaled to fit the ratings of the PV described in table \ref{tab:solar_pv}. Fig \ref{fig:PV_PROFILE_8} shows the eight-day PV profile used.

\begin{figure}[!ht]
    \centering
    \includegraphics[width = \linewidth]{figs/PV_PROFILE.png}
    \caption{Eight day PV profile}
    \label{fig:PV_PROFILE_8}
\end{figure}


