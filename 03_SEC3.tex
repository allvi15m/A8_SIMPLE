By defining the solution space with a combination of nodes and edges the optimization problem can be formulated as a graph search problem. At the start, the starting node is determined by the current status of the system. Then the following nodes and edges are generated using the forecasted data available. A discrete set of endpoints are set as the goals. The A*  algorithm runs for all the goal nodes and the node with the lowest path cost is selected as the best end node. The shortest graph search path to that node is selected as the optimum path.

The following are the main steps of the search algorithm,
\begin{enumerate}
\item \textbf{Set goal node:} Set the current goal node as the first component of the predefined list of end nodes. 
%2
\item \textbf{Create open list from start node:} Expand the start node and create an open list with all the child nodes of the start node. The open list sorts it's members in a priority queue based on the cost. The the start node is added to the closed list.
%3
\item \textbf{Select and expand best node from open list:} The node with the lowest cost is selected from the nodes within the open list. The Selected node is expanded and the children of the selected node is added to the open list. The expanded node is then added to the closed list.
%4
\item \textbf{Repeat 3 until goal node is reached}
%5
\item \textbf{Construct shortest path to goal node}: After reaching the goal node the shortest path to the goal is constructed from the closed list by retracing the parent nodes from the goal node to the start node.
%6
\item \textbf{Calculate and record total cost:} The total cost for the shortest path is calculated and the path is recorded with the total cost in a list named \textit{path list}.
%7
\item \textbf{Set new goal}: The open and closed list are cleared. The next component in the list of end nodes is selected as the goal node.
%8
\item \textbf{Repeat 2-7 for all components of the end nodes list}
%9
\item \textbf{Select the path with the lowest cost in the path list as the optimum path.}
\end{enumerate}

The EMS recalculates the optimum path using the search algorithm at every time step based on the updated information. The system status is assumed to be constant between time steps.

%\subsection{Cost Calculation}
The cost of going from a parent node to a child node is calculated by combining the real cost of getting to that child node and heuristic cost of getting to the goal from that child node. The real cost of going from a parent node $p$ at time $T=t$ to a child node $c$ at time $T=t+\Delta T$ is denoted as $C_{actual}(pc)$. It is calculated according to equation (\ref{eq:C_actual}).

\begin{equation}
\label{eq:C_actual}
    C_{actual}(pc) =  C_{ESS}(pc)+C_{GRID}(t)+C_{best}(p)
\end{equation}

Here, $\Delta T$ represents the time between two time steps. $C_{actual}(pc)$ represent the total cost of going to the child node $c$ from parent node $p$. $C_{ESS}(pc)$ represent cost of energy storage to go from parent node $p$ to child node $c$. $C_{GRID}(t)$ is the cost of using the grid between time $T=t$ and time $T=t+\Delta T$. $C_{best}(p)$ represent the best or least cost to get to the parent node $p$. $C_{ESS}(pc)$ is calculated according to equation \ref{eq:C_ESS}.

\begin{equation}
\label{eq:C_ESS}
C_{ESS}(pc) = |(SOC_p - SOC_c)|*ESS_{CAP}*R_{ESS} 
\end{equation}
Here, $SOC_p$ and $SOC_c$ represent the state of charge at parent and child node. $ESS_{CAP}$ represent the total energy capacity of the energy storage. And $R_{ESS}$ is the $\$/kWh$ cost of using the energy storage. $C_{GRID}(t)$ is calculated according to equation \ref{eq:C_GRID}.

\begin{equation}
\label{eq:C_GRID}
C_{GRID}(t) = 
\begin{cases}
   E_{GRID}(t)*RTP(t),& \text{if } E_{GRID}(t)\geq 0\\
    E_{GRID}(t)*SP(t),& \text{if }  E_{GRID}(t) < 0
\end{cases}
\end{equation}

Here, $E_{GRID}(t)$ is the energy drawn from the grid between time $T=t$ and time $T=t+\Delta T$. $RTP(t)$ is the real time price between $t$ and $t+\Delta T$. $SP(t)$ is the price the utility is willing to pay the consumer for selling power between $t$ and $t+\Delta T$.

The heuristic cost is calculated by assuming that whichever source has a smaller cost during a time step will supply the total demand of that time step. The heuristic cost of a node at time $T = t$ is calculated according to equation \ref{eq:C_H}.


\begin{equation}
\label{eq:C_H}
C_H(t) = \sum_{n=t}^{end} D(t)*R_{best}(t)
\end{equation}

Here, $C_H(t)$ represent the heuristic cost. $D(t)$ is the demand between time $T = t$ and time $T = t+\Delta T$. $R_{best}(t)$ is the source with the smaller cost which is calculated according to equation \ref{eq:R_best}

\begin{equation}
\label{eq:R_best}
R_{best}(t) = 
\begin{cases}
    R_{ESS},& \text{if } RTP(t)\geq R_{ESS}\\
    RTP(t),              & \text{otherwise}
\end{cases}
\end{equation}

After calculating the actual cost  $C_{actual}(pc)$ and heuristic cost $C_H(t)$ the total cost is calculated by adding  $C_{actual}(pc)$ and $C_H(t)$.

