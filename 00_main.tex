
%use mybib.bib for bibliography. bibtex is used for bibliography
\documentclass[journal]{IEEEtran}
\usepackage[utf8]{inputenc}
\usepackage{graphicx}
\usepackage{cite}
\usepackage{longtable}
\usepackage{amsmath}
\usepackage{multirow}
\usepackage{multicol}
%\usepackage{wrapfig}
\usepackage{float}
%\usepackage[section]{placeins}
%\usepackage{subcaption}
\usepackage{array}
\usepackage[export]{adjustbox}
\usepackage{tabu}
\usepackage{listings}
\usepackage{siunitx}
\usepackage{siunitx}
\usepackage[usenames, dvipsnames]{color}


%%%%%%%%%%%%%%%%%%%%%%%%%%%%%%%%%%%%%%%%%%%%%%%%%%%%%


\ifCLASSINFOpdf

\else

\fi


\hyphenation{op-tical net-works semi-conduc-tor}


\begin{document}

\title{A* Based Optimum Control of Enterprise Level Energy Storage Based on Forecasting of PV, Load and Real-Time price of Energy}

\author{ Alvi~Newaz,~\IEEEmembership{Student Member,~IEEE,}
Juan~Ospina,~\IEEEmembership{Student Member,~IEEE and}
     Omar~Faruque,~\IEEEmembership{Senior Member,~IEEE}
        }% <-this % stops a space





\maketitle                                                               

\begin{abstract}
This paper proposes a novel energy storage management (ESM) solution designed to optimize the use of grid-connected distributed energy resources (DERs) based on forecasted values of generation and load. It will also take into account real-time energy prices. The control strategy aims to control the charging and discharging of the energy storage so that the total cost of using energy to serve the local load is minimized.
The proposed solution generated a graph from the energy storage status and available data and used A* search algorithm to find the optimum path through the grid. Seven day test results are presented and compared with two scenarios. The first scenario compares a sampling-based model predictive control (SBMPC) based solution which considers net metering with the proposed A* based ESM. The other scenario compares the A* based EMS with two base test cases using pricing scheme obtained from Pacific Gas and Electric (PG\&E) and New York Independent System Operator (NYISO) considering a sell back price of energy which is different from the price of buying energy from the grid. The results show substantial cost savings over the compared schemes.


\end{abstract}


\begin{IEEEkeywords}
Distributed Energy Resources, Energy Storage System, Energy Storage Management, Real-Time Pricing, A* Search Algorithm.
\end{IEEEkeywords}


\IEEEpeerreviewmaketitle



\section{Introduction}
Grid-connected distributed energy resources (DERs), have seen exponential growth in the last few years. The unprecedented growth of DERs has been driven by the decentralization of power systems and the growth in deployment of distributed generation (DG) and distributed storage (DS) systems by both utility companies and consumers. Progressively, energy storage systems are becoming more competitive. Companies are starting to heavily invest in the development of lithium-ion batteries, thermal storage, and other types of DS systems with the objective of decreasing energy costs and stabilizing the distribution system. 

DG and DS systems have the potential of becoming the cornerstone of the future smart grid. Nonetheless, these systems are still not ready for a harmonious integration to the grid due to their lack of control and intermittent nature of DG \cite{denholm2016path}. In the US, most of the DG and DS systems are being deployed under two basic operating principles: 1) to help in the reduction of metered load through net metering programs, and 2) to sell real power generation to utility companies through power purchase agreements (PPAs). These unsophisticated methods of transactions with the energy market limit the optimal utilization of DER systems. The lack of control and dispatchability of these systems can lead to power being lost due to DG diversion. That's why, in order to maximize the DG penetration and effective utilization of available resources, DERs need to have a more sophisticated control approach that dynamically leverages all the resources available to the system while serving the load in an economical, reliable, and safe way. Such optimal control of the energy resources will translate into direct benefits to both utility companies and regular consumers. Due to the intermittent characteristics and constraints present in some DER systems, such as solar and wind, optimal dispatch control has become a challenging optimization problem tackled by many researchers.

Energy management algorithms which deal with both variable DG and real-time pricing can be divided into two main categories, offline and real-time algorithms \cite{rt_shi_2017}. The offline optimization algorithms usually optimize the energy management for a predetermined time horizon based on prediction data. Some research has been done considering perfect prediction of the future \cite{Off_1,off_2,off_3,off_4}. But in reality, there are uncertainties in the prediction horizon. To solve the problem of reliance on perfect prediction some offline algorithms, try to model the uncertainties in a day ahead scheduling and try to develop optimum schedules taking into account the probable uncertainties. \cite{ous_1,ous_2} use mixed integer linear programming for optimizing DERs. To take into account the uncertainties in the prediction horizon \cite{ous_2} generates multiple solutions based on different scenarios. \cite{ous_1} also produces different risk averse and risk-neutral options. In both cases, the scenarios are generated stochastically. The number of scenarios required to be generated to properly capture the uncertainties are large in number. This makes these process computationally extensive. The researchers in \cite{ous_3} formulate the optimization problem based on the worst-case scenario. The solution method considers the most cost from the grid and the least benefits from using local DG resources. The uncertainty of the grid is expressed as set points in Taguchi’s orthogonal array. Then search strategy based on Taguchi’s orthogonal array is used to find the best possible solution cases. Using the worst case to formulate the problem makes the solution robust for all situations. Although this method provides a robust solution always taking into account worst scenario makes the solution suboptimal. 

There have been advances recently in designing real-time algorithms that optimize the long-term cost of the energy management system taking into account the intermittent nature of the loads and DGs in the grid. Researchers in \cite{rt1} use an aggregator to minimize the long-term cost of the grid. The solution used Lyapunov optimization to solve the energy management problem in real-time. The method proposed does not consider the actual architecture of the grid and optimizes considering all the components are connected to the same bus. This makes it impractical due to the actual grid having line constraints that cannot be violated.\cite{rt2} formulates the problem as a real-time one leader N-follower Stackelberg game and shows an efficient approach designed to optimize the system based on a demand response scheme. The optimization is performed by creating a virtual electricity trading market where the facility (leader) produces virtual prices of electricity and the resources (followers) compete to buy and sell electricity. The researchers show the existence of unique Stackelberg equilibrium for optimum use of each resource. In \cite{rt3}, researchers propose a multi-period AC optimal power flow taking into account the uncertainties of the solar and wind resources to ensure reliable solutions for the distribution system operators. In the first stage the algorithm contacts with all the available resources to determine the upper and lower bounds of generation throughout the next day. The upper and lower bounds of the different units are termed as flexibility in the paper. In the second stage which is the real-time operation, the algorithm optimizes in real-time taking into account the flexibility of the available resources.  \cite{rt4} proposes to solve the problem using an online energy management system in real-time using Lyapunov optimization taking into account the physical constraints of the system.
The drawbacks of these methods are that most of them do not take into account the actual distribution system architecture while formulating the problem. Only \cite{rt4} considers system architecture while optimizing in real-time. But a lack of prediction based elements makes these methods optimal for short-term costs but suboptimal for properly optimizing long-term operating cost.

From the discussion thus far it is evident that there is a need for a real-time energy management solution(EMS) that can optimize the long-term operating cost of a system containing  DG and energy storage (ES). This paper presents an optimal real-time control strategy for such systems. The proposed control strategy takes into account the forecasted states of the available resources and real-time price to optimize the total cost of energy usage. the rest of the paper is structured as follows. Section II introduces the formulation of the problem as a graph search. Section III introduces the A* search algorithm which is used to solve the optimization problem. Section IV discusses the test system. Section V presents results and section VI presents the conclusions.



\section{Problem formulation}
\label{formulation}
The proposed solution technique is designed to be implemented at the PCC(point of common coupling) of a microgrid containing distributed generation and energy storage. The objective is to optimize the use of energy storage under different pricing schemes taking advantage of DG and load forecasting. Figure \ref{fig:F1_CA} shows the top level architecture of the control strategy. As seen in the figure the BMS(battery management system) will optimize the power use from the DG plant, the utility grid and the ES. It will take in the RTP prediction, load prediction and PV prediction and generate optimum battery charge and discharge references based on the current and forecasted data.

\begin{figure}[!ht]
    \centering
    \includegraphics[width = \linewidth]{figs/EMS_FIG.png}
    \caption{Controller top level architecture}
    \label{fig:F1_CA}
\end{figure}

The solution has to take into account the current status of the system as well as the future forecasted state of the system to determine the most optimum time to charge and discharge the energy storage. To find the most cost optimum solution based on the current system status and future forecasts the optimization problem is formulated as a graph search problem. To represent the solution space of the problem as a graph the state of charge (SOC) of the ESS is discretized. Also the time until the prediction horizon is also represented in discrete steps. Figure \ref{fig:F1_Dis} represents a simple example of the discretized solution space.

\begin{figure}[!ht]
    \centering
    \includegraphics[width = \linewidth]{figs/F1_1_Dis.png}
    \caption{Discretizing solution space}
    \label{fig:F1_Dis}
\end{figure}
Here, the horizontal axis represents time and vertical axis represents discrete states of charge for the energy storage. In this scenario it is assumed that the algorithm recalculates the solution every 15 minutes based on available data. The SOC of the ESS is discretized in steps of 20\% and the SOC is limited between 80\% and 20\%. It is also assumed that the ESS can discharge a maximum of 40\% of it's maximum charge and charge a maximum of 20\% of it's charge in a 15 minute time step. Taking these features into consideration a directed graph is constructed in figure \ref{fig:F1_Dis}. The colored boxes represent nodes on the graph. The numbers inside the boxes represent the SOC of the ESS at that node. The arrows from the boxes represent all the possible states the ESS can be in in the next time step according to the constraints mentioned before. The arrows are treated as edges of the graph nodes. In this case the edges are unidirectional. The cost of the edges for a node at time $T = n$ are calculated using equation (\ref{eq:1}).
\begin{equation}
\label{eq:1}
    C_{EDGE} = C_{ESS}+C_{GRID}(n)
\end{equation}
Where,
$$
C_{ESS} = E_{ESS}*R_{ESS} 
$$
$$
E_{ESS} = (SOC_p - SOC_c)*ESS_{CAP}
$$
$$
C_{GRID}(n) = (P_{GRID}(n)*\Delta T + E_{ESS})*RTP(n)
$$

Here, $C_{EDGE}(n)$, $C_{ESS}(n)$ and $C_{GRID}(n)$ represent the cost of the Edge, cost of energy storage system and the cost of using the grid energy. $SOC_p and SOC_c$ are the states of charge of the parent and child node respectively. The parent node is the node the edge originates from and the child node is the node where the edge ends. $ESS_{CAP}$ is the total capacity of the energy storage. $R_{ESS}$ represent the \$/kwh cost of using the ESS. $P_{GRID}(n)$ is the power required from the grind at the $n^{th}$ time step. $\Delta T$ is the difference in time between the $n^{th}$ and $(n+1)^{th}$ time step. $RTP(n)$ represents the real time price of energy at the $n^{th}$ time step. 

\section{A* based energy management system} \label{A*}
\subsection{A* search algorithm}
A* is a computer algorithm which is widely used to solve graph search problems. It determines the most efficient path between multiple nodes in a graph. A* is an informed search algorithm. This means it searches between all the possible paths to the goal and finds the path incurring the least cost. To do this it considers the paths which appear to have the least cost to get to the goal first. Starting from a specific node it explores the graph step by step depending on the cost of going from one node to the next. The algorithm selects the note to explore based on a combination of the actual cost to get to the node and a heuristic cost that estimates the cost from the current node to the goal. The process to calculate heuristic cost is problem-specific. For the algorithm to work properly the heuristic cost has to be less than or equal to the actual cost of getting to the goal node. This means the heuristic cost function should never overestimate the cost for reaching the goal. The algorithm works by calculating the combined actual and heuristic cost for all the neighboring nodes of the starting node and puts them into a priority queue called the open list. Then it selects the node with the least cost and expands that node to get the cost of its neighbors. The expanded node is taken out of the priority queue and put in another list called the closed list. then the algorithm continues to expand the open list always selecting the node with the least cost and adding it to the closed list. The algorithm stops when the goal node is in the open list and it has the minimum cost in the open list. Then the algorithm retraces it's path through the closed list to find the optimum path.

\subsection{Implementation}
By defining the solution space with a combination of nodes and edges the optimization problem can be formulated as a graph search problem. At the start, the starting node is determined by the current status of the system. Then the following nodes and edges are generated using the forecasted data available. A discrete set of endpoints are set as the goals. The A*  algorithm runs for all the goal nodes and the node with the lowest path cost is selected as the best end node. The shortest graph search path to that node is selected as the optimum path.

The following are the main steps of the search algorithm,

\hline
\textbf{Algorithm 1:} A* Algorithm
\hline

\begin{algorithmic}[1]
\label{al:1}
\Function{A*}{$start, goal$}
\State $closedSet := \{ \} $ \Comment{Set of evaluated nodes}
\State $openSet := {start}$ \Comment{Set of already discovered nodes. Initially,  the start node is discovered.}

\State $bestParrent := \{ \}$ \Comment{Best previous state. Initially empty.}

\State $gCost := \infty $ \Comment{The cost for travelling to any node from start node. Initialized as infinity for all unevaluated nodes.}

\State $hCost := \infty $ \Comment{The cost for travelling to goal node from any node.}

\State $fCost := gCost + hCost$

\State $gCost[start] := 0$ 

\State $hCost[start] := heuristic\_estimate (start, goal)$ \Comment{Heuristic cost estimation function between two nodes.}

\State $fCost[start] := gCost[start]+hCost[start]$

\While{$openSet$ is not empty}

    \State $current := $ node in $openSet$ with lowest $fCost$
    
    \If{current = goal}
      \State \REturn $ \ reconstruct\_path (bestParrent, current)$
      \State \Break $ $
    \EndIf
    
    \State $openSet.Remove(Current)$
    \State $closedSet.Add(Current)$
    \For{each neighbor of current }
                
    \EndFor

\EndWhile
    
\EndFunction
\end{algorithmic}


The EMS recalculates the optimum path using the search algorithm at every time step based on the updated information. The system status is assumed to be constant between time steps.

%\subsection{Cost Calculation}
The cost of going from a parent node to a child node is calculated by combining the real cost of getting to that child node and heuristic cost of getting to the goal from that child node. The real cost of going from a parent node $p$ at time $T=t$ to a child node $c$ at time $T=t+\Delta T$ is denoted as $C_{actual}(pc)$. It is calculated according to equation (\ref{eq:C_actual}).

\begin{equation}
\label{eq:C_actual}
    C_{actual}(pc) =  C_{ESS}(pc)+C_{GRID}(t)+C_{best}(p)
\end{equation}

Here, $\Delta T$ represents the time between two time steps. $C_{actual}(pc)$ represent the total cost of going to the child node $c$ from parent node $p$. $C_{ESS}(pc)$ represent cost of energy storage to go from parent node $p$ to child node $c$. $C_{GRID}(t)$ is the cost of using the grid between time $T=t$ and time $T=t+\Delta T$. $C_{best}(p)$ represent the best or least cost to get to the parent node $p$. $C_{ESS}(pc)$ is calculated according to equation \ref{eq:C_ESS}.

\begin{equation}
\label{eq:C_ESS}
C_{ESS}(pc) = |(SOC_p - SOC_c)|*ESS_{CAP}*R_{ESS} 
\end{equation}
Here, $SOC_p$ and $SOC_c$ represent the state of charge at parent and child node. $ESS_{CAP}$ represent the total energy capacity of the energy storage. And $R_{ESS}$ is the $\$/kWh$ cost of using the energy storage. $C_{GRID}(t)$ is calculated according to equation \ref{eq:C_GRID}.

\begin{equation}
\label{eq:C_GRID}
C_{GRID}(t) = 
\begin{cases}
   E_{GRID}(t)*RTP(t),& \text{if } E_{GRID}(t)\geq 0\\
    E_{GRID}(t)*SP(t),& \text{if }  E_{GRID}(t) < 0
\end{cases}
\end{equation}

Here, $E_{GRID}(t)$ is the energy drawn from the grid between time $T=t$ and time $T=t+\Delta T$. $RTP(t)$ is the real time price between $t$ and $t+\Delta T$. $SP(t)$ is the price the utility is willing to pay the consumer for selling power between $t$ and $t+\Delta T$.

The heuristic cost is calculated by assuming that whichever source has a smaller cost during a time step will supply the total demand of that time step. The heuristic cost of a node at time $T = t$ is calculated according to equation \ref{eq:C_H}.


\begin{equation}
\label{eq:C_H}
C_H(t) = \sum_{n=t}^{end} D(t)*R_{best}(t)
\end{equation}

Here, $C_H(t)$ represent the heuristic cost. $D(t)$ is the demand between time $T = t$ and time $T = t+\Delta T$. $R_{best}(t)$ is the source with the smaller cost which is calculated according to equation \ref{eq:R_best}

\begin{equation}
\label{eq:R_best}
R_{best}(t) = 
\begin{cases}
    R_{ESS},& \text{if } RTP(t)\geq R_{ESS}\\
    RTP(t),              & \text{otherwise}
\end{cases}
\end{equation}

After calculating the actual cost  $C_{actual}(pc)$ and heuristic cost $C_H(t)$ the total cost is calculated by adding  $C_{actual}(pc)$ and $C_H(t)$.





\section{Test system} \label{sys}
For testing the algorithm described in section \ref{A*} a Florida feeder available in the SUNGRIN project \cite{SUNGRIN} is used. Fig. \ref{fig:simulation_grid} shows the whole system used for testing the algorithm. The yellow borders marks the section of the feeder modified to construct a microgrid similar to the one shown in Fig \ref{fig:system_arch}. The PV and load inside the microgrid is modeled using the load and solar data collected form the SUNGRIN project. An energy storage is included to construct the microgrid. Table \ref{tab:solar_pv} shows the physical parameters of the PV plant and its inverter. The physical parameters of the energy storage used are shown in table \ref{tab:es}. The levelized cost of energy (LCOE)  of the energy storage system $R_{ESS}$ is calculated using equation \ref{eq:R_ESS}.

\begin{figure}[!ht]
    \centering
    \includegraphics[width = \linewidth]{figs/simulation_grid.png}
    \caption{Simulated system}
    \label{fig:simulation_grid}
\end{figure}


\begin{equation}
\label{eq:R_ESS}
R_{ESS} = \dfrac{ES_{tot}}{Cyc\cdot ES_{Cap}\cdot DoD\cdot \eta_{r}},
\end{equation}

%%%%%%%%PV%%%%%%%%%%%%%%%%%%%%%%%%%%%%%%%%%%%
\begin{table}[htb]
%\normalsize
%\renewcommand{\arraystretch}{1}
\caption{PV System Specifications}
\label{tab:solar_pv}
\centering
    \begin{tabular}{ | l | p{3cm} | }
    \hline
    \textbf{PV System Parameters} & \textbf{Value} \\ \hline
    PV Panels Rating (\(P_{PV}\)) & 875 kW  \\ \hline
    Inverter Rating (\(S_{PV}\)) & 900 kVA \\ \hline
    Power Factor Range (\(pf_{PV}\)) & 0.8-1.0  \\ \hline
    Max. Reactive Power (\(\overline{Q_{PV}}\)) & 540 kVAR \\ \hline
    Min. Reactive Power (\(\underline{Q_{PV}}\)) & -540 kVAR \\ \hline
    LCOE (\(r_{PV}\)) & 2.51 c/kWh \\ \hline
    \end{tabular}
    \begin{tabular}{l}
    \end{tabular}
\end{table}
%%%%%%%%PV%%%%%%%%%%%%%%%%%%%%%%%%%%%%%%%%%%%


%%%%%%%%ES%%%%%%%%%%%%%%%%%%%%%%%%%%%%%%%%%%%
\begin{table}[htb]
%\normalsize
%\renewcommand{\arraystretch}{1}
\caption{Energy Storage (ES) System Specifications}
\label{tab:es}
\centering
    \begin{tabular}{ | l | p{3cm} | p{3cm} | }
    \hline
    \textbf{ES System Parameters} & \textbf{Value} \\ \hline
    ES Rating (\(P_{ES}\)) & 750 kW  \\ \hline
    Inverter Rating (\(S_{ES}\)) & 750 kVA \\ \hline
    Max. State of Charge  (\(\overline{SOC_{ES}}\)) & 2190 kWh \\ \hline
    Min. State of Charge  (\(\underline{SOC_{ES}}\)) & 219 kWh \\ \hline
    Power Factor Range (\(pf_{ES}\)) & 0.8-1.0  \\ \hline
    Max. Reactive Power (\(\overline{Q_{ES}}\)) & 450 kVAR \\ \hline
    Min. Reactive Power (\(\underline{Q_{ES}}\)) & -450 kVAR \\ \hline
    LCOE (\(r_{ES}\)) & 12.3 c/kWh \\ \hline
    \end{tabular}
\end{table}
%%%%%%%%ES%%%%%%%%%%%%%%%%%%%%%%%%%%%%%%%%%%%

Fig \ref{fig:LOAD_PROFILE_8} shows the load profile of the system for eight days with average, minimum and maximum loads.

\begin{figure}[!ht]
    \centering
    \includegraphics[width = \linewidth]{figs/loadprofile.png}
    \caption{Eight day load profile}
    \label{fig:LOAD_PROFILE_8}
\end{figure}

To generate the RTP profile Locational Based Marginal Pricing (LBMP) data is collected from the New York Independent System Operator (NYISO) \cite{NYISO2017}. The collected LBMPO is combined with the time of use price available at Tallahassee to generate a RTP for the test case. Fig \ref{fig:RTP_PROFILE_8} shows the real-time price profile used for the test system. The system is also tested against time of use pricing scheme available from PG\&E \cite{pgne}. Fig \ref{fig:PGNE_PRICE} shows the price profile acquired from PG\&E.

\begin{figure}[!ht]
    \centering
    \includegraphics[width = \linewidth]{figs/rtp_8days.png}
    \caption{Eight day RTP profile NYISO}
    \label{fig:RTP_PROFILE_8}
\end{figure}

\begin{figure}[!ht]
    \centering
    \includegraphics[width = \linewidth]{figs/PGNE_PRICE.png}
    \caption{Eight day RTP profile PG\&E}
    \label{fig:PGNE_PRICE}
\end{figure}

The PV profile used in the system is collected from the SUNGRIN project and scaled to fit the ratings of the PV described in table \ref{tab:solar_pv}. Fig \ref{fig:PV_PROFILE_8} shows the eight-day PV profile used.

\begin{figure}[!ht]
    \centering
    \includegraphics[width = \linewidth]{figs/PV_PROFILE.png}
    \caption{Eight day PV profile}
    \label{fig:PV_PROFILE_8}
\end{figure}




\section{Simulation and results}
The performances of the A* based ESM is tested for two different test scenarios. In scenario 1 the system is tested against a Sampling-Based Model Predictive Control (SBMPC) scenerio (CITE DOCTDER JOURNAL). In scenario 2 the ESM is tested against two simple base test cases using PGNE test data.

\subsection{Comparison with SBMPC}
 Model predictive control (MPC) uses a predictive model to optimize a cost function while enforcing constraints on the sytem inputs and outputs and is widely applied in industrial process control \cite{qin1997overview}. Typically, industrial MPC is implemented for linear models, but the use of nonlinear models allows better performance over a wider operating range \cite{berber2012nonlinear}. SBMPC \cite{dunlap2011book}, \cite{reese2016graph} is an MPC method that uses a receding horizon along with the optimization algorithm, Sampling-Based Model Predictive Optimization (SBMPO), which samples the input of the predictive model to compute a  graph tree with nodes and branches. Figure \ref{fig:EMS_7_7days} shows the response of the EMS for 7 days.
 
 \begin{figure}[!ht]
    \centering
    \includegraphics[width = \linewidth]{figs/Plot100_7.png}
    \caption{EMS response}
    \label{fig:EMS_7_7days}
\end{figure}

\subsection{PG\&E peak day pricing test result}

Figure \ref{fig:EMS_7_PGNE} shows the PGNE test results for the 7 days.

\begin{figure}[!ht]
    \centering
    \includegraphics[width = \linewidth]{figs/PGNE_PEAK_100_7_24.png}
    \caption{EMS response}
    \label{fig:EMS_7_PGNE}
\end{figure}



\section{Conclusions}
This paper proposed a novel EMS aimed to determine the optimal use of ESS. The EMS scheme is designed to optimize the charging and discharging of a grid connected energy storage by formulating the problem as a graph search and solving the graph search using A* search algorithm. The A* based ESM scheme is tested against a SBMPO based control strategy which uses a similar concept of graph search to solve the problem.
The testing shows a 1\% cost saving compared to the SBMPO scheme. The proposed EMS is also tested using realistic data collected from the SUNGRIN project, NYISO, and PG\&E. It also considers a sell back price different from the buying price of energy. The ESM shows a cost saving of 6.93\% to 41.79\% compared to the base test cases for this scenario. The results show that the proposed method has the capability to adapt to various changing price profile and system status and provide a solution that reduces cost significantly. Future research will focus on developing an EMS capable of controlling multiple ESS and additional  DERs taking into account system constraints.




\bibliographystyle{IEEEtran}
\bibliography{mybib.bib}

\ifCLASSOPTIONcaptionsoff
  \newpage
\fi

\end{document}


