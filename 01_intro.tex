In the last few years, there has been significant growth in grid-connected distributed energy resources (DERs). This has lead to an increased deployment of distributed generation (DG) and distributed storage (DS) systems. Moreover, companies are starting to heavily invest in the competitive energy storage system market with a goal of decreasing the cost of energy storage. Although a significant amount of DG and DS are being added to the distribution grid they are still not ready for a seamless integration to the grid due to the lack of control and intermittent nature of non-dispatchable DGs \cite{denholm2016path}. In the US most DG and DS systems are deployed to either help reduce the metered load through net metering program or to sell power to the utility through power purchase agreements (PPAs). These methods fail to fully utilize the potential of DG combined with DS systems due to the lack of proper control and dispatchability. That is why a sophisticated energy management solution is required to maximize the use of available DERs. Such an energy management solution will be able to dynamically optimize the use of all available DS in order to serve the load the most economical and safe way. This will benefit both utility companies and regular consumers. Due to constraints and intermittent nature of some DG systems, such as wind and solar, the optimum management of DS combined with a DG is a difficult problem to solve. The most common approaches found in the current literature are as follows. Some researchers formulate the problem as a linear programming (LP) or mixed integer linear programming (MILP) model \cite{lp73, lp74, lp75}. Authors in \cite{pso80, pso81} present an energy management solution based on particle swarm optimization for a microgrid containing wind turbines and energy storage (ES) system. Other researchers propose crow-search \cite{csa87} and ant colony optimization \cite{aco84}. There has also been model predictive control (MPC) based control approaches for managing ES in a microgrid \cite{energymanajaboulay,mpcmorstyn}. Others have also proposed genetic algorithm based solution to optimize the ES operation in a microgrid \cite{ga76, ga77}. Most of these approaches only consider the current status of the system. But there are some critical factors like real-time price (RTP), forecasted load and generation profiles. These factors can be used to potentially increase the performance of energy storage management (ESM) operation. These information can be used to find an optimum solution based on both current and probable future states of the system as opposed to a solution relying only on the currently available data.

To incorporate the available predicted data into ESM solutions off-line day ahead planning models has also been proposed. In these methods the ESM solution is determined based on Monte Carlo simulations \cite{6872821,7010943,6839110}. These solutions are computationally intensive and require time for planning the day ahead. This property makes them unsuitable for real-time implementation and very reliant on the accuracy of the predicted data. Taking all these factors into consideration this paper proposes an optimal ESM solution that takes into account the current and future states of the system and optimizes the cost of operating the ES using the A* search algorithm.

The paper is organized as follows. Section II discusses how the ESM problem is formulated as a graph search problem. Section III introduces the A* graph search algorithm used to find the optimal solution and how it is used to find the solution. Section IV discusses the test system used in simulation to verify the performance of the graph search based ESM algorithm. Section V discusses the results obtained from the simulation and Section VI presents conclusions.