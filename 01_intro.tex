% The conventional power grid had an unpredictable and varying load profile. To counter this the traditional grid used the dispatchability of traditional power plants and spinning reserves to maintain reliable power flow. With the increased integration of distributed kd.kj


To counter the variability of the grid, the current power grid uses the dispatchability of traditional generation and spinning reserves. With integration of distributed generation(DG) specially non-dispatchable DG the variability of the grid is changing rapidly \cite{denholm2016path}. 

Energy management algorithms which deal with both variable DG and real-time pricing can be divided into two main categories, offline and real-time algorithms \cite{rt_shi_2017}.The offline optimization algorithms usually optimize the energy management for a predetermined time horizon based on prediction data. Some research has been done considering perfect prediction of the future \cite{Off_1,off_2,off_3,off_4}. But in reality there are uncertainties in the prediction horizon. To solve the problem of reliance on perfect prediction some offline algorithms, try to model the uncertainties in day ahead scheduling and try to develop optimum schedules taking into account the probable uncertainties. \cite{ous_1,ous_2} use mixed integer linear programming for optimizing DERs. To take into account the uncertainties in the prediction horizon \cite{ous_2} generates multiple solutions based on different scenarios. \cite{ous_1} also produces different risk averse and risk neutral options. In both cases the scenarios are generated stochastically. The number of scenarios required to be generated to properly capture the uncertainties are large in number. This makes these process computationally extensive. The researchers in \cite{ous_3} formulate the optimization problem based on the worst case scenario. The solution method considers the most cost from the grid and the least benefits from using local DG resources. The uncertainty of the grid is expressed as set points in Taguchi’s orthogonal array. Then search strategy based on Taguchi’s orthogonal array is used to find the best possible solution cases. Using the worst case to formulate the problem makes the solution robust for all situations. Although this method provides a robust solution always taking into account worst scenario makes the solution suboptimal. 

There has been advances recently in designing real-time algorithms that optimize the long-term cost of the energy management system taking into account the intermittent nature of the loads and DGs in the grid. 


(Juan's part)


This paper presents an optimal control strategy for energy storage system (ESS) considering DG integration. The proposed control strategy takes into account the forecasted states of the available resources and real-time price to optimize the total cost of energy usage. This paper is structured as follows. Section II introduces the formulation of the problem as a graph search. Section III introduces the A* search algorithm which is used to solve the optimization problem. Section IV discusses the test system and results. Finally section V presents the conclusion.

